\documentclass[paper=letter, fontsize=11pt]{scrartcl} % Letter paper and 11pt font size

\usepackage{amstext, amsmath, amssymb, graphicx}
\usepackage[T1]{fontenc} % Use 8-bit encoding that has 256 glyphs
\usepackage[english]{babel} % English language/hyphenation
\usepackage{amsmath,amsfonts,amsthm} % Math packages

\usepackage{fancyhdr} % Custom headers and footers
\pagestyle{fancyplain} % Makes all pages in the document conform to the custom headers and footers
\fancyhead{} % No page header
\fancyfoot[L]{} % Empty left footer
\fancyfoot[C]{} % Empty center footer
\fancyfoot[R]{\thepage} % Page numbering for right footer
\renewcommand{\headrulewidth}{0pt} % Remove header underlines
\renewcommand{\footrulewidth}{0pt} % Remove footer underlines
\setlength{\headheight}{13.6pt} % Customize the height of the header
\setlength\parindent{0pt} % Removes all indentation from paragraphs

%----------------------------------------------------------------------------------------
%	TITLE SECTION
%----------------------------------------------------------------------------------------

\newcommand{\horrule}[1]{\rule{\linewidth}{#1}} % Create horizontal rule command with 1 argument of height

\title{	
\normalfont \normalsize 
\textsc{San Francisco State University} \\ [25pt]
\horrule{0.5pt} \\[0.4cm] % Thin top horizontal rule
\huge MATH 490 Assignment 5  \\ % The assignment title
\horrule{2pt} \\[0.5cm] % Thick bottom horizontal rule
}

\author{Omar Sandoval}

\date{\normalsize\today}

\begin{document}

\maketitle

%----------------------------------------------------------------------------------------
%	PROBLEM 3.13
%----------------------------------------------------------------------------------------
\textbf{3.13} Access the horseshoe crab data of Table 3.2 at www.stat.ufl.edu/~aa/intro-c
da/appendix.html. \\
\textbf{a.} Using $x=$ weight and $Y =$ number of satellites, fit a Poisson loglinear
model. Report the prediction equation. \\
\textbf{b.} Estimate the mean of $Y$ for female crabs of average weight $2.44$kg. \\
\textbf{c.} Use $\hat{\beta}$ to describe the weight effect. Construct a $95$ percent conf
idence interval for $\beta$ and for the multiplicative effect of a 1 kg increase. \\
\textbf{d.} Conduct a Wald test of the hypothesis that the mean of $Y$ is independent of 
weight. Interpret. \\
\textbf{e.} Conduct a likelihood-ratio test about the weight effect. Interpret. \\

%----------------------------------------------------------------------------------------
%	PROBLEM 3.14
%----------------------------------------------------------------------------------------
\textbf{3.14} Refer to the previous exercise. Allow overdispersion by fitting the negativ
e binomial loglinear model. \\
\textbf{a.} Report the prediction equation and the estimate of the dispersion parameter a
nd its $SE$. Is there evidence that this model gives a better fit than the Poisson model?
\\
\textbf{b.} Constuct a 95 percent confidence interval for $\beta$. Compare it with the on
e in (c) in the previous exercise. Interpret, and explain why the interval is wider with 
the negative binomial model. \\

%----------------------------------------------------------------------------------------
%	PROBLEM 3.18
%----------------------------------------------------------------------------------------
\textbf{3.18} Table 3.8 lists total attendance (in thousands) and the total number of arr
ests in a season for soccer teams in the Second Division of the British football league.
\\
\textbf{a.} Let $Y$ denote the number of arrests for a team with total attendance $t$. Ex
plain why the model $E(Y) = \mu t$ might be plausible. Show that it has alternative form 
$log[E(Y)/t] = \alpha$, where $\alpha = log(\mu)$, and express this model with an offset 
term. \\
\textbf{b.} Assuming Poisson sampling, fit the model. Report and interpret $\hat{\mu}$.\\
\textbf{c.} Plot arrests against attendance, and overlay the prediction equation. Use res
iduals to identify teams that had a much larger or smaller than expected number of arrest
s. \\
\textbf{d.} Now fit the model $log[E(Y)/t] = \alpha$ by assuming a negative binomial dist
ribution. Compare $\hat{\alpha}$ and its \textit{SE} to what you got in (a). Base on this 
information and the estimate of the dispersion parameter and its \textit{SE}, does the Po
isson assumption seem appropriate? \\

%----------------------------------------------------------------------------------------
%	PROBLEM 3.20
%----------------------------------------------------------------------------------------
\textbf{3.20} \textit{Do not do.}
\end{document}
