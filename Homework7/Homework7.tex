\documentclass[paper=letter, fontsize=11pt]{scrartcl} % Letter paper and 11pt font size

\usepackage{amstext, amsmath, amssymb, graphicx}
\usepackage[T1]{fontenc} % Use 8-bit encoding that has 256 glyphs
\usepackage[english]{babel} % English language/hyphenation
\usepackage{amsmath,amsfonts,amsthm} % Math packages

\usepackage{fancyhdr} % Custom headers and footers
\pagestyle{fancyplain} % Makes all pages in the document conform to the custom headers and footers
\fancyhead{} % No page header
\fancyfoot[L]{} % Empty left footer
\fancyfoot[C]{} % Empty center footer
\fancyfoot[R]{\thepage} % Page numbering for right footer
\renewcommand{\headrulewidth}{0pt} % Remove header underlines
\renewcommand{\footrulewidth}{0pt} % Remove footer underlines
\setlength{\headheight}{13.6pt} % Customize the height of the header
\setlength\parindent{0pt} % Removes all indentation from paragraphs

%----------------------------------------------------------------------------------------
%	TITLE SECTION
%----------------------------------------------------------------------------------------

\newcommand{\horrule}[1]{\rule{\linewidth}{#1}} % Create horizontal rule command with 1 argument of height

\title{	
\normalfont \normalsize 
\textsc{San Francisco State University} \\ [25pt]
\horrule{0.5pt} \\[0.4cm] % Thin top horizontal rule
\huge MATH 490 Assignment 7  \\ % The assignment title
\horrule{2pt} \\[0.5cm] % Thick bottom horizontal rule
}

\author{Omar Sandoval}

\date{\normalsize\today}

\begin{document}

\maketitle

%----------------------------------------------------------------------------------------
%	PROBLEM 4.19
%----------------------------------------------------------------------------------------
\textbf{4.19} A sample of subjects were asked their opinion about current laws 
legalizing abortion (support, oppose). For the explanatory variables gender (female,
male), religious affiliation (Protestant, Catholic, Jewish), and political party
affiliation (Democrat, Republican, Independent), the model for the probability
$\pi$ of supporting legalized abortion, \\

\begin{center}
    $\text{logit} (\pi) = \alpha + \beta_h^G + \beta_i^R + \beta_j^P$
\end{center}

has reported parameter estimates (setting the parameter for the last category
of a variable equal to 0.0) $\hat{\alpha} = -0.11, \hat{\beta_1^G} = 0.16, 
\hat{\beta_2^G} = 0.0, \hat{\beta_1^R} = -0.57, \hat{\beta_2^R} = -0.66, 
\hat{\beta_3^R} = 0.0, \hat{\beta_1^P} = 0.84, \hat{\beta_2^P} = -1.67, 
\hat{\beta_3^P} = 0.0$ \\

\textbf{a.} Interpret how the odds of supporting legalized abortion depend on gender. \\
To find the how odds of supporting legalized abortion depend on gender, we need to 
control religion and political party. Thus, the difference between females and males
turns out to be $\hat{\beta_1^G} - \hat{\beta_2^G} = 0.16$. \\
Odds of females supporting legalized abortion are $e^{.16} = 1.17$. So, the odds of 
females supporting legalized abortion are 1.17 times higher than the odds for a male.
\\

\textbf{b.} Find the estimated probability of supporting legalized abortion for (i) male
Catholic Republicans and (ii) female Jewish Democrats. \\
We use the following equation, $\hat{\pi} = e^{\text{logit}}/1+e^{\text{logit}}$. \\
(i) $\hat{\pi} = \dfrac{e^{-0.11-0.66-1.67}}{1+e^{-0.11-0.66-1.67}} = .08$ \\
For male Catholic republicans, probability that they support legalized abortion is 0.08. \\
(ii) $\hat{\pi} = \dfrac{e^{-0.11+0.16+.84}}{1+e^{-0.11+0.16+.84}} = .71$ \\
For female Jewish democrats, probability that they support legalized abortion is .71. \\
\\

\textbf{c.} If we defined parameters such that the first category of a variable has value
0, then what would $\hat{\beta_2^G}$ equal? Show then how to obtain the odds ratio that
describes the conditional effect of gender. \\
The odds ration is invariant under permutation of rows and columns, but sign will have 
to change. So, $e^{\beta_2^G} = \frac{1}{e^{\beta_1^G}} = e^{-\beta_1^G}$. Thus, 
$\beta_2^G = -\beta_1^G = -0.16$ \\
If the first category of a variable had value 0, then $\hat{\beta_2^G} = -.16$. The odds
ratio would not change. The odds ratio would still be $e^{.16} = 1.17$.
\\

\textbf{d.} If we defined parameters such that they sum to 0 across the categories of a
variable, then what would $\hat{\beta_1^G}$ and $\hat{\beta_2^G}$ equal? Show then how
to obtain the odds ratio that describes the conditional effect of gender.
\\
We can find the odds ratio for the conditional effect of gender like so; \\
$\dfrac{e^{\alpha}e^{\beta_2^G}}{e^{\alpha}e^{\beta_1^G}} = e^{-0.16} = 0.85$
for Male, Female. \\
$\dfrac{e^{\alpha}e^{\beta_1^G}}{e^{\alpha}e^{\beta_2^G}} = e^{0.16} = 1.17$
for Female, Male.
\\

%----------------------------------------------------------------------------------------
%	PROBLEM 4.21
%----------------------------------------------------------------------------------------
\textbf{4.21} In a study designed to evaluate whether an educational program makes
sexually active adolescents more likely to obtain condoms, adolescents were randomly
assigned to two experimental groups. The educational program, involving a
lecture and videotape about transmission of the HIV virus, was provided to one
group but not the other. In logistic regression models, factors observed to influence
a teenager to obtain condoms were gender, socioeconomic status, lifetime
number of partners, and the experimental group. Table 4.17 summarizes study
results. \\

\textbf{a.} Interpret the odds ratio and the related confidence interval for the effect
of group. \\
The odds ratio tells us that the odds of obtaining condoms for the educated group 
are 4.04 times the odds for the non-educated group.
\\

\textbf{b.} Find the parameter estimates for the fitted model, using (1, 0) indicator
variables for the first three predictors. Based on the corresponding confidence
interval for the log odds ratio, determine the standard error for the
group effect. \\
The equation is logit$(\hat{\pi}) = \hat{\alpha} + 1.40i + .32j + 1.76k + 1.17l$
Here, $i$ is 0 for non-educated and 1 for educated. $j$ is is 0 for females and 1 for
males. $k$ is 0 for low SES and 1 for high SES. $l$ is the lifetime number of partners.
\\
The log odds ratio of 1.40 has CI (0.16, 2.63). The interval goes from $1.40 \pm 1.96(SE)$
So it's clear that the standard error is 0.63.
\\

\textbf{c.} Explain why either the estimate of 1.38 for the odds ratio for gender or the
corresponding confidence interval is incorrect. Show that, if the reported
interval is correct, then 1.38 is actually the log odds ratio, and the estimated
odds ratio equals 3.98. \\
\textit{Not exactly sure what they're asking here.}
\\

%----------------------------------------------------------------------------------------
%	PROBLEM 4.23
%----------------------------------------------------------------------------------------
\textbf{4.23} Table 4.18 shows estimated effects for a fitted logistic regression model 
with squamous cell esophageal cancer (1 = yes, 0 = no) as the response variable
Y . Smoking status (S) equals 1 for at least one pack per day and 0 otherwise,
alcohol consumption (A) equals the average number of alcoholic drinks
consumed per day, and race (R) equals 1 for blacks and 0 for whites. \\

\textbf{a.} To describe the race-by-smoking interaction, construct the prediction 
equation when R = 1 and again when R = 0. Find the fitted Y - S conditional odds
ratio for each case. Similarly, construct the prediction equation when S = 1
and again when S = 0. Find the fitted Y - R conditional odds ratio for each
case. Note that, for each association, the coefficient of the cross-product
term is the difference between the log odds ratios at the two fixed levels for
the other variable. \\

When $R = 1$, logit$(\hat{\pi}) = (-7.0 + 0.3) + 0.10A + (1.2 + 0.2)S = -6.7 + .10A + 1.4S$
\\
The Y - S conditional odds ratio for blacks is $e^{1.4} = 4.0552$. 
\\
For $R = 0$, logit$(\hat{\pi}) = (-7.0 + 1.2) + 0.10A + (.30 + 0.2)R = -5.8 + .10A + .5R$
\\
The Y - R conditional odds ratio for smokers with consumption of at least one pack per day
is $e^{0.3} = 1.3499$
\\
For $S = 0$, logit$(\hat{\pi}) = -7.0 + 0.10A + .30R$
\\
The Y - R odds ratio for people with consumption less than one pack per day is
$e^{0.3} = 1.3499$
\\

\textbf{b.} In Table 4.18, explain what the coefficients of R and S represent, for the
coding as given above. What hypotheses do the P-values refer to for these
variables? \\

The coefficient R represents the conditional log odds ratio between race and cancer for
people with consumption that is less than one pack per day. \\
The coefficient S represents the conditional log odds ratio between gender and cancer 
for whites. \\
The P-value < 0.01 for smoking represents the result of the test that the log odds ratio Y - S
for whites is 0. P-value = 0.02 for race represents the result of the test where the log
odds ratio between Y - R for smokers where consumption of at least one pack per day is 0.

\textbf{c.} Suppose the model also contained an A × R interaction term, with coefficient
0.04. In the prediction equation, show that this represents the
difference between the effect of A for blacks and for whites.
\\

%----------------------------------------------------------------------------------------
%	PROBLEM 4.24
%----------------------------------------------------------------------------------------
\textbf{4.24}
\begin{center}
	\textbf{Already completed 4.24 problem in HW 6}
\end{center}

%----------------------------------------------------------------------------------------
%	PROBLEM 4.28
%----------------------------------------------------------------------------------------
\textbf{4.28} For recent General Social Survey data, a prediction equation relating Y =
whether attended college (1 = yes) to \textit{x} = family income (thousands of dollars,
using scores for grouped categories), \textit{m} = whether mother attended
college (1 = yes, 0 = no), \textit{f} = whether father attended college (1 = yes,
0 = no), was logit[$\hat{P}(Y = 1)] = −1.90 + 0.02x + 0.82m + 1.33f$. To summarize
the cumulative effect of the predictors, report the range of $\hat{\pi}$ values
between their lowest levels $(x = 0.5, m = 0, f = 0)$ and their highest levels
$(x = 130, m = 1, f = 1)$.
\\
$\hat{\pi}$ increases from $\dfrac{e^{-1.89}}{1+e^{-1.89}} = .13 \text{ to } 0.95$
What can we say from this???
\\

%----------------------------------------------------------------------------------------
%	PROBLEM 4.37
%----------------------------------------------------------------------------------------
\textbf{4.37} For data from Florida on Y = whether someone convicted of multiple murders
receives the death penalty (1 = yes, 0 = no), the prediction equation is
logit($\hat{\pi}$) $= −2.06 + .87d − 2.40v$, where $d$ and $v$ are defendant's race and
victims' race (1 = black, 0 = white). The following are true-false questions
based on the prediction equation. \\

\textbf{a.} The estimated probability of the death penalty is lowest when the defendant
is white and victims are black. \\

True.
$\pi(d = 0, v = 1) = \dfrac{e^{-2.06-2.40}}{1+e^{-2.06-2.40}} = 0.01143$
\\
\textbf{b.} Controlling for victims' race, the estimated odds of the death penalty for
white defendants equal 0.87 times the estimated odds for black defendants.
If we instead let $d = 1$ for white defendants and 0 for black defendants, the
estimated coefficient of $d$ would be $1/0.87 = 1.15$ instead of $0.87$. \\

False. \\
If we control victims race, the estimated odds of death penalty for whites is 
$e^{0.87}$ times the estimated odds for black defendants.
\\

\textbf{c.} The lack of an interaction term means that the estimated odds ratio between
the death penalty outcome and defendant's race is the same for each category
of victims' race. \\

False. \\
We only know that new odds ratio will equal the reciprocal of the old odds ratio. As we 
saw in problem 4.19(c).
\\

\textbf{d.} The intercept term $−2.06$ is the estimated probability of the death penalty
when the defendant and victims were white (i.e., $d = v = 0$). \\

False. \\
Probability MUST be between 0 and 1 and thus cannot be -2.06.

\textbf{e.} If there were 500 cases with white victims and defendants, then the model
fitted count (i.e., estimated expected frequency) for the number who receive
the death penalty equals $500e^{−2.06}/(1 + e^{−2.06})$. \\

True. \\

\end{document}
